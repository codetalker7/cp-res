\documentclass[12pt,a4paper]{amsart}
\usepackage{mathtools}% http://ctan.org/pkg/mathtools
\usepackage{amsfonts}
\usepackage{amsthm}
\usepackage[section]{placeins}
\usepackage{relsize}
\usepackage{amsmath}
\usepackage{amssymb}
\usepackage{amscd}
\usepackage{caption} 
\usepackage[utf8]{inputenc}
\usepackage{setspace}
\usepackage[mathscr]{eucal}
\usepackage{indentfirst}
\usepackage{graphicx}
\graphicspath{{./images/}}

%preamble for including inkscape figures
\usepackage{import}
\usepackage{xifthen}
\usepackage{pdfpages}
\usepackage{transparent}

\usepackage{graphics}
\usepackage{tikz}
\usepackage{pict2e}
\usepackage{epic}
\usepackage{mathrsfs}
\numberwithin{equation}{section}
\usepackage[margin=2.8cm]{geometry}
\usepackage{epstopdf} 

%the following code is for a good table of contents for amsart. Finding it was not easy. for more information, check out the following two links
%https://tex.stackexchange.com/questions/51760/table-of-contents-with-indents-and-dots
%https://tex.stackexchange.com/questions/58324/adding-indents-to-the-toc-without-adding-dotted-lines-between-sections
\makeatletter
\def\@tocline#1#2#3#4#5#6#7{\relax
  \ifnum #1>\c@tocdepth % then omit
  \else
    \par \addpenalty\@secpenalty\addvspace{#2}%
    \begingroup \hyphenpenalty\@M
    \@ifempty{#4}{%
      \@tempdima\csname r@tocindent\number#1\endcsname\relax
    }{%
      \@tempdima#4\relax
    }%
    \parindent\z@ \leftskip#3\relax \advance\leftskip\@tempdima\relax
    \rightskip\@pnumwidth plus4em \parfillskip-\@pnumwidth
    #5\leavevmode\hskip-\@tempdima
      \ifcase #1
       \or\or \hskip 1em \or \hskip 2em \else \hskip 3em \fi%
      #6\nobreak\relax
    \dotfill\hbox to\@pnumwidth{\@tocpagenum{#7}}\par
    \nobreak
    \endgroup
  \fi}
\makeatother
%table of contents code finished

%packages for adding code. use \begin{lstlisting}[language=C++]\end{lstlisting} environment to write code. In text, can also use \verb|something|
\usepackage{listings}
\usepackage{color}
\definecolor{codegreen}{rgb}{0,0.6,0}
\definecolor{codegray}{rgb}{0.5,0.5,0.5}
\definecolor{codepurple}{rgb}{0.58,0,0.82}
\definecolor{backcolour}{rgb}{0.95,0.95,0.92}
\lstdefinestyle{mystyle}{
    backgroundcolor=\color{backcolour},   
    commentstyle=\color{codegreen},
    keywordstyle=\color{magenta},
    numberstyle=\tiny\color{codegray},
    stringstyle=\color{codepurple},
    basicstyle=\ttfamily\footnotesize,
    breakatwhitespace=false,         
    breaklines=true,                 
    captionpos=b,                    
    keepspaces=true,                 
    numbers=left,                    
    numbersep=5pt,                  
    showspaces=false,                
    showstringspaces=false,
    showtabs=false,                  
    tabsize=2
}
\lstset{style=mystyle}

%all about hyperlinks. include this package as the final package to avoid any problems
\usepackage{hyperref}
\hypersetup{
    colorlinks=true,
    linkcolor=blue,
    filecolor=magenta,      
    urlcolor=cyan,
}
    
%to create a new numbered environment, use the syntax
%\newtheorem{name}{display}
%to start renumbering say at every section, add a [section] in the end. To have the same numbering for two environments, do as I did with Lemma below, i.e add [Th] in between.

%let the thoerems have italic text
\newtheorem{Th}{Theorem}[section]
\newtheorem{Lemma}[Th]{Lemma}
\newtheorem{Cor}{Corollary}[Th]
\newtheorem{Prop}[Th]{Proposition}

%definitions, examples, remarks and exercises will have normal text
\theoremstyle{definition}
\newtheorem{Def}{Definition}[section]
\newtheorem{Ex}{Example}[section]
\newtheorem{Remark}{Remark}[Th]
\newtheorem{Exercise}{Exercise}[section]

%using an unnumbered section for solutions
\newtheorem*{solution}{Solution}

%use the proof environment for writing proofs. I'm using a black square for QED. 
\renewcommand\qedsymbol{$\blacksquare$}


\newcommand{\N}{\mathbb{N}}
\newcommand{\Z}{\mathbb{Z}}
\newcommand{\Ztwo}{\mathbb{Z}_2}
\newcommand{\Q}{\mathbb{Q}}
\newcommand{\R}{\mathbb{R}}
\newcommand{\CC}{\mathbb{C}}
\newcommand{\E}{\mathbb{E}}
\renewcommand{\P}{\mathbb{P}}
    
%package for fonts
\usepackage{fontspec}
%main font
%\setmainfont{Chivo}

\begin{document}

\title{Standard Template Library in C++}
\author{Siddhant Chaudhary}
\date{December 2020}

\maketitle
    
\begin{abstract}
This is a reference sheet for the STL in C++.
\end{abstract}
    
\tableofcontents

\subsection{Iterators} In all STL classes, we use \textit{iterators} instead of \textit{pointers}. Iterators are used to point to elements in an STL class; for instance, if \verb|v| is a vector, then the iterator \verb|v.begin()| points to the first element of the vector \verb|v|. Iterators work just like pointers; so the first element of the vector \verb|v| will be \verb|*(v.begin())|. To store the iterator in a variable, we can do something like \verb|stl_class::iterator p|. For instance, in maps of the type \verb|map <int , int>|, we can do the following.  
\begin{lstlisting}[language=C++]
    map <int , int> mymap;
    map <int , int>::iterator p; p = mymap.begin();
    cout << *(p).first; //prints the key of the first element in the map
\end{lstlisting}
A similar line can be used to declare iterators for other STL classes.

\subsection{Vector} Consider \verb|vector <int> v|. Then the following are defined.
\begin{enumerate}
    \item \verb|v.assign(no_of_elements , val); //initialisation of v|
    \item \verb|v.size(); //returns the size|
    \item \verb|v.empty(); // returns whether v is empty|
    \item \verb|v.front(); // first element|
    \item \verb|v.back(); // last element|
    \item \verb|v.push_back(x); // push x in v|
    \item \verb|v.pop_back(); // pop last element|
    \item \verb|v.begin(); // iterator to beginning. This is an iterator|
    \item \verb|v.end(); // iterator to end. NOT the last element. This is an iterator|
    \item \verb|v.rbegin(); // reverse iterator to beginning, i.e iterator to last element|
    \item \verb|v.rend(); // reverse iterator to beginning. NOT the first element|
\end{enumerate}
\begin{Remark}
    Consider the iterators \verb|v.end()| and \verb|v.rend()|. In C++, the last element of the vector \verb|v| will be succeeded by a \textit{theoretical last element}. So, the element \verb|*(v.end())| will actually \textit{not} be the last element of \verb|v|; instead, the last element will be \verb|*(v.end() - 1)|. A similar thing holds for the reverse iterator.
\end{Remark}

\subsection{Queue} Consider \verb|queue <int> myqueue|. Then the following are defined. 
\begin{enumerate}
    \item \verb|myqueue.empty()|
    \item \verb|myqueue.size()|
    \item \verb|myqueue.front()|
    \item \verb|myqueue.back()|
    \item \verb|myqueue.push(x)|
    \item \verb|myqueue.pop()|
\end{enumerate}

\subsection{Deque} Consider \verb|deque <int> mydeque|. Then the following are defined; 
\begin{enumerate}
    \item \verb|mydeque.assign(no_of_elements , val); //initialisation of mydeque|
    \item \verb|mydeque.size(); //returns the size|
    \item \verb|mydeque.empty();|
    \item \verb|mydeque.front();|
    \item \verb|mydeque.back();|
    \item \verb|mydeque.push_back(x);|
    \item \verb|mydeque.pop_back();|
    \item \verb|mydeque.push_front(x);|
    \item \verb|mydeque.pop_front();|
    \item \verb|mydeque.begin();|
    \item \verb|mydeque.end();|
    \item \verb|mydeque.rbegin();|
    \item \verb|mydeque.rend();|
\end{enumerate}

\subsection{Set and Multiset} These are implemented as balanced BSTs in C++, and their operations take time $O(\log n)$. Both of these are ordered structures. The operations for \verb|set| and \verb|multiset| are the same. Consider \verb|set <int> myset|. 
\begin{enumerate}
    \item \verb|myset.begin()|
    \item \verb|myset.end()|
    \item \verb|myset.rbegin()|
    \item \verb|myset.rend()|
    \item \verb|myset.empty()|
    \item \verb|myset.size()|
    \item \verb|myset.insert(x)|
    \item \verb|myset.erase(position)| or \verb|myset.erase(value)|. \verb|position| must be an iterator and this erasing takes constant time. \verb|value| must be a value contained in the set, and this erasing takes logarithmic time.
    \item \verb|myset.find(value)|. Get the iterator (pointer) where the value \verb|value| is stored inside the set. If \verb|value| is not in the set, simply return \verb|set.end()|. Takes logarithmic time.
\end{enumerate}
If you want to pass a general comparison function to a \verb|set| or \verb|multiset|, then you can do something like this. 

\begin{lstlisting}[language=C++]
    struct my_structure{
        ll x , y , z;
    };

    class compare_function{
        public:
            bool operator()(const my_structure &a , const my_structure &b){
                //sort on the basis of the first coordinate
                return a.x < b.x;
            }
    };
    
    //initialise the set or multiset like this 
    set <my_structure , compare_function> myset;
\end{lstlisting}

\subsection{Maps and Unordered Maps} A set that contains key-value pairs, and the keys must be \textit{distinct}. \verb|map| is an ordered set, while \verb|unordered_map| is not ordered; \verb|map| uses the BST data structure, and hence operations are of the order $O(\log n)$. For \verb|unordered_map|, the operations are $O(1)$ on average, since these use hashing. Let \verb|map <int , int> mymap| be a definition. 
\begin{enumerate}
    \item \verb|mymap.begin()|
    \item \verb|mymap.end()|
    \item \verb|mymap.rbegin()|
    \item \verb|mymap.rend()|
    \item \verb|mymap.empty()|
    \item \verb|mymap.size()|
    \item \verb|mymap.insert(pair<int , int>(x , y)) //insert the key-value pair (x , y)|
    \item \verb|mymap.erase(position)| or \verb|myset.erase(key)|. Erasing by \verb|position| takes constant time, while erasing by \verb|key| takes logarithmic time.
    \item \verb|mymap.find(key)|. Get the iterator to the pair with key attribute equal to \verb|key|.
\end{enumerate}
\noindent As usual, we can pass a custom key comparison function to a map. So you can do something like this. 

\begin{lstlisting}[language=C++]
    struct key_type{
        ll x , y , z;
    };

    class compare_function{
        public:
            bool operator()(const key_type &a , const key_type &b){
                return a.x < b.x;
            }
    };
    
    //initialise the map as follows 
    map <key_type , mapped_type , compare_function> mymap;
\end{lstlisting}

\subsection{Priority Queue} Look at this example. 
    \begin{lstlisting}[language=C++]  
        struct Node {
            int x , y , value;
        };
        class compare{
            public: 
                int operator()(Node &a, Node &b){
                    return a.value > b.value;
                }
        };
        priority_queue <Node , vector<Node> , compare> myqueue;
        Node mynode;
        myqueue.push(Node);
        myqueue.pop();
    \end{lstlisting}
First, only look at the last four lines. We have defined \verb|myqueue| to be a priority queue, which contains values of type \verb|Node|, and the comparison function is \verb|compare|. The class \verb|compare| is just used to say what the comparison function does. You can literally use any comparison function of your choice! 

\begin{Remark}
    In this case, we will have a min-priority queue. To have a max-priority queue, just invert the inequality in the compare class. 
\end{Remark}

\noindent The following are defined. 
\begin{enumerate}
    \item \verb|myqueue.empty()|
    \item \verb|myqueue.size()|
    \item \verb|myqueue.top() // top element|
    \item \verb|myqueue.push(x) // x must be of type Node|
    \item \verb|myqueue.pop() // pop the top element|
\end{enumerate}

\subsection{Bitsets} We will use the following for reference \textcolor{red}{(this is also given in the main document)}.

Declare a \verb|bitset| like this: \verb|bitset <size> mybitset|. Given two bitsets \verb|a| and \verb|b|, the following are defined. 
\begin{enumerate}
    \item \verb|a.count() //the number of set bits in a|
    \item \verb|a[i] //access the i^{th} bit|
    \item \verb|a.size() //size of the bitset|
    \item \verb|a.test(pos) //true if bit at position pos is 1, false otherwise|
    \item \verb|a.any() //test if any bit is set|
    \item \verb|a.none() //test if no bit is set|
    \item \verb|a.all() //test if all bits are set|
    \item \verb|a.set() // set all the bits|
    \item \verb|a.set(pos)// set the bit at position pos|
    \item \verb|a.set(pos , val) //set bit at position pos to val|
    \item \verb|a.reset() //reset all the bits, i.e all the bits to 0|
    \item \verb|a.reset(pos) // reset the bit at position pos|
    \item \verb|a.flip() //flip all the bits|
    \item \verb|a.flip(pos) //flip the bit at position pos|
\end{enumerate}

The good thing about bitsets is that you can apply the usual bitwise operations to them. We can also construct a bitset from integers or strings, as follows. 
\begin{lstlisting}[language=C++]
    bitset<100> a(17);
    bitset<100> b("1010");
\end{lstlisting}

\section{Strings in C++} 
\noindent These are specially defined in the STL. See the \href{http://www.cplusplus.com/reference/string/string/}{documentation} for more information. 

\subsection{scanf and printf with strings} The functions \verb|scanf| and \verb|printf| do not support any C++ class. In particular, we cannot take input using this function to a \verb|string| object. For example, the following code will not make sense.
\begin{lstlisting}[language=C++]
    string temp;
    scanf("%s" , &temp); //will not work
\end{lstlisting}
Instead, the following can be used. 
\begin{lstlisting}[language=C++]
    char temp1[100];
    scanf("%s" , &temp1);
    string temp = temp1; //this works
\end{lstlisting}

\subsection{Comparing strings} NEVER try to compare two strings using the \verb|==| operator. For instance, both of the below snippets are not encouraged. 
\begin{lstlisting}[language=C++]
    char temp1[100]; char temp2[100];
    if (temp1 == temp2){ \\don't do this
        ...
    }
\end{lstlisting}

\begin{lstlisting}[language=C++]
    string temp1; string temp2;
    if (temp1 == temp2){ \\don't do this
        ...
    }
\end{lstlisting}
Instead, convert the given strings to a \verb|string| object, and then use the \verb|std::string.compare| function. 

\subsection{scanf return values are useful} The \verb|scanf| function returns the number of items of the argument list successfully filled (see \url{https://www.cplusplus.com/reference/cstdio/scanf/}). This means the return value of \verb|scanf| can be very useful in parsing input. For instance, consider the problem \href{https://onlinejudge.org/index.php?option=com_onlinejudge&Itemid=8&category=623&page=show_problem&problem=166}{UVa230}, in which we have to consider a specific input format. In order to input the whole library of books, we can do something as follows.
\begin{lstlisting}[language=C++]
    //taking the library books
    char title[81], author[81];
        
    while(scanf("%s by %s", title, author) == 2){
        string tit = title, auth = author;
        //do something
    }
\end{lstlisting}

\subsection{memcpy and memset with strings} Using \verb|memcpy| and \verb|memset| with strings is very useful. For example, do something like this. 
\begin{lstlisting}[language=C++]
    char my_string[80];
    memset(my_string, 'a', 80); //my_string becomes aaaa..aa (80 a's)
    char another_string[80];
    memcpy(another_string, my_string, 80) //copying my_string to another_string
\end{lstlisting}

\subsection{String parsing with scanf} In many problems, especially ICPC problems, one has to parse the input in a very specific way. To do this, \verb|scanf| comes in handy, and the \verb|cout| way of input is highly discouraged (as it will make parsin difficult to code). The basic format specifier of \verb|scanf| is as follows. 
\begin{verbatim}
    %[*][width][length]specifier
\end{verbatim}
\verb|specifier| can take the following values. 
\begin{enumerate}
    \item \verb|d|: integer.
    \item \verb|x|: hexadecmial integer.
    \item \verb|f|: floating point number.
    \item \verb|c|: character.
    \item \verb|s|: any number of non-whitespace characters.
    \item \verb|p|: pointer.
    \item \verb|[characters]|: any number of characters specified within the brackets.
    \item \verb|[^characters]|: any number of characters \textit{not within} the characters specified within the brackets.
    \item \verb|%%|: a single \verb|%| sign. 
\end{enumerate}
Here are the meaning of the optional specifiers. 
\begin{enumerate}
    \item The optional specifier \verb|*| indicates that the data is to be read but ignored.
    \item \verb|width| specifies the maximum number of characters to be read in the current reading operation.
    \item \verb|length| alters the expected type of the storage pointed by the corresponding argument. \verb|length| has to be one of \verb|hh, h, l, ll, j, z, t, L|. 
\end{enumerate}
Some cool examples of string parsing are given below. 
\begin{lstlisting}[language=C++]
    scanf("\"%s\"\n"); //reads a string  without whitespace within quotation marks
    scanf("\"[^\"]\"\n"); //reads a string (with white space possibly) without the " character within quotation marks
    scanf("[^\n]\n"); //reads a line, without the \n character
\end{lstlisting}


\section{Policy Based Data Structures}
\subsection{Reference blog} I used the following link to learn about this data structure:  \url{https://codeforces.com/blog/entry/11080}.

\subsection{Order Statistics Tree} This data structure is also known as an \textit{ordered statistics tree}. For competitive programming, the best way to initialise this structure has the following template. 
\begin{lstlisting}[language=C++]6
	tree <
		key_type,
		mapped_type,
		key_comparison_functor,
		rb_tree_tag, //tree will be implemented as a red-black tree
		tree_order_statistics_node_update
	>	mytree;
\end{lstlisting}
Above, \verb|key_type| is the type for keys, \verb|mapped_type| is the type of the mapped values and \verb|key_comparison_functor| is the functor which will be used to compare keys; it can be defined in a very similar fashion as with \verb|priority_queue|, with a slight difference: the method \verb|operator()| of the comparison class must be declared to be \verb|const|, otherwise the compiler will complain. An example of a typical \verb|key_comparison_functor| is as follows. 
\begin{lstlisting}[language=C++]
	class compare{
		public:
			//declare method to be const
			bool operator()(const pair<ll, ll> &a, const pair<ll, ll> &b) const{
				//only compare based on first coordinates
				return a.first < b.first;
			}
	};
	typedef tree<
		pair<ll, ll>,
		null_type,
		compare,
		rb_tree_tag,
		tree_order_statistics_node_update
	> ordered set;

\end{lstlisting} 

If you want just a set, you can put \verb|mapped_type = null_type|. The following methods are supported.
\begin{enumerate}
	\item Methods supported by \verb|set|.
	\item \verb|find_by_order(k)|: returns iterator to the \verb|k|th largest element (counting from 0). 
	\item \verb|order_of_key(k)|: returns the number of items in the structure with key strictly less than key \verb|k|. 
\end{enumerate}

\subsection{Example Problem} To practice using this data structure, try this problem out: \url{https://codeforces.com/contest/1579/problem/E2}.

\section{Random Number Generation}
\subsection{Reference blog} Check out this reference blog to know more: \url{https://codeforces.com/blog/entry/61587}. 

\subsection{Initialising the Mersenne Twister Generator in C++} To initialise the generator, we will use the current system time as the seed (using a fixed seed is not recommended for competitive programming, as it might results in hacks and someone challenging your code). 
\begin{lstlisting}[language=C++]
    mt19937 rng(chrono::steady_clock::now().time_since_epoch().count());
\end{lstlisting}
Use \verb|mt19937_64| to generate 64-bit numbers. 

\subsection{The uniform integer distribution} To generate a random integer in a given range $[a , b]$ uniformly, you can do the following. 
\begin{lstlisting}[language=C++]
    <type> x = uniform_int_distribution<type>(a , b)(rng)
\end{lstlisting}
Here \verb|type| must be an integer type (\verb|int|, \verb|long long int| etc). \verb|rng| is the generator that we initialised before. 

\subsection{Random shuffle} To randomly shuffle a vector, you can do something like this. 
\begin{lstlisting}[language=C++]
    vector <long long int> v;
    shuffle(v.begin(), v.end(), rng);
\end{lstlisting}
\end{document}